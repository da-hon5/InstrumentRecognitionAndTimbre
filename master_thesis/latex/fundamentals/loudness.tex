% **************************************************************************************************
% **************************************************************************************************
\newsection{Loudness}{fundamentals:loudness}
According to~\cite[p. 133]{moore2012introduction}, \enquote{Loudness is defined as that attribute of auditory sensation in terms of which sounds can be ordered on a scale extending from quiet to loud.} How loud an audio signal is perceived mainly depends on sound pressure level (SPL), frequency content and its duration. Initial research on loudness perception focused on pure tones. In 1933, Fletcher and Munson~\cite{fletcher1933loudness} published their famous equal-loudness contours as shown in Fig.~\ref{fig:equal-loudness-curves}. These curves were obtained through a listening test in which participants had to adjust the intensity level of a \SI{1}{\kilo\hertz} reference tone until it sounded equally loud as various sine tones of different frequencies. The experiment showed that the sensitivity of the human ear is highest at approximately \SI{4}{\kilo\hertz} and decreases in both directions of the frequency axis.\\
\figc{width=0.65\textwidth}{\pwd/figs/equal-loudness-curves}{Equal-loudness contours of pure tones. Image borrowed from~\cite{fletcher1933loudness}.}{equal-loudness-curves}

Since loudness is a subjective quantity, it can only be approximated by objective measures. In 2004, an evaluation of 12 loudness meters, submitted by researchers and private companies, was conducted~\cite{soulodre2004evaluation}. The winner was a simple algorithm called \textit{Leq(RLB)} -- consisting of a frequency weighting function followed by \gls{rms} calculation -- proposed by the author of the paper. In  2015, the \textit{international telecommunication union (ITU)} published an efficient algorithm to measure subjective loudness of mono, stereo and multichannel audio signals~\cite{itu2015recommendation}. Based on \textit{Leq(RLB)}, this algorithm is the recommended way to determine the perceived loudness of digital audio signals in applications such as television or radio. In Fig.~\ref{fig:itu-loudness-algorithm}, its basic principle is depicted. In the following, a rough overview of the algorithm's building blocks is given. The interested reader is referred to~\cite{itu2015recommendation} for more details.\\

The proposed algorithm comprises four main stages. Initially, a filter stage, termed \textit{K-weighting}, is applied. It is a combination of the RLB weighting curve from \textit{Leq(RLB)} and a high shelf filter to account for the acoustic effects of the human head. Subsequently, the input signal's power (mean square) is computed over overlapping \SI{400}{\milli\second} blocks. Afterwards, the channels are multiplied with individual weights $G$ and summed up. Finally, a sophisticated gate is applied which discards blocks below a certain threshold. The purpose of this gate is to stop the measurement if the signal drops below a threshold. Otherwise the computed average loudness would be distorted due to silent sections in the audio. The unit of the obtained loudness value is \textit{\gls{lkfs}}. An increment of \SI{1}{\decibel} in the level of a signal leads to a loudness increase of 1~\gls{lkfs}. \textit{\Gls{lufs}}, an alternative unit proposed in~\cite{ebu2020recommendation}, is equivalent to \gls{lkfs}. Keep in mind that in psychoacoustics loudness is measured in \textit{sone}. However, sone is not a suitable measure to compare the loudness of digital audio signals as it depends on the analog amplification of the signal, characteristics of the loudspeakers etc.\\
\figc{width=1.0\textwidth}{\pwd/figs/itu-r_loudness_algorithm}{Multichannel loudness algorithm proposed by the ITU. Image borrowed from~\cite{itu2015recommendation}.}{itu-loudness-algorithm}

In this work, all audio data is converted to mono, hence only the single channel case of the algorithm in Fig.~\ref{fig:itu-loudness-algorithm} is utilized. A Python package called \textit{pyloudnorm}~\cite{steinmetz2021pyloudnorm} was used for the implementation of loudness measurements in our code. Note that the algorithm described above is just an approximation, albeit a good one, of the perceived loudness. It was developed for broadcasting scenarios where low computational cost is desirable. More sophisticated models try to emulate specific characteristics of the human ear; such as critical bands or the transmission of sound through the ear. However, the physiological and psychological processes of loudness perception are still not fully understood~\cite[p. 139]{moore2012introduction}. For a detailed discussion of all the effects involved in loudness perception, we recommend the psychoacoustics \enquote{bibles} \cite{zwicker2013psychoacoustics} and \cite{moore2012introduction}.
