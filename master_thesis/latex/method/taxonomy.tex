% **************************************************************************************************
% **************************************************************************************************
\newsection{Instrument Taxonomy}{method:taxonomy}
Since the three multi-track datasets used in this work exhibit different class structures, it is necessary to construct a unified instrument taxonomy. Moreover, we want this taxonomy to be hierarchical, hoping that our \glspl{dnn} can leverage the additional information to learn a broader concept of musical instruments. Similar approaches were already taken several times in the deep learning literature and were successfully implemented for \gls{ir} applications~\cite{garcia2021leveraging}. As a general rule for training \glspl{dnn}, a sufficient number of examples for every class is required. In other words, classes with insufficient examples have to be discarded, which basically means that some information gets lost. However, our multi-track datasets are highly imbalanced and for certain instruments only few examples are available. To utilize these underrepresented classes for training after all, a two-level hierarchy was defined. It is inspired by the \textit{Hornbostel-Sachs}~\cite{hornbostel1914systematik} system, which classifies musical instruments based on their underlying sound production mechanisms. As the physical principle of an instrument strongly correlates with its sound, such a taxonomy seems appropriate for our use case. The first level of the proposed taxonomy, as shown in Fig.~\ref{fig:taxonomy}, represents eight instrument families: voice, percussion, bowed string, plucked string, woodwind, brass, key and synth. The second level embodies seven specific instruments for which abundant data was available: singer, drums, violin, electric guitar, acoustic guitar, electric bass and piano.\\

Hereafter, peculiarities of some instrument families are discussed. First of all, note that string instruments, which are usually played with a bow, can also be plucked. This technique -- also known as pizzicato -- is particularly common for the contrabass in jazz. We accepted this unavoidable error and assigned all instruments, which are bowed most of the time, to the \textit{bowed strings} family. Furthermore, the \textit{key} family is quite diverse, as the only similarity within this group is that all members are played with a keyboard. However, the sound production mechanisms of, for instance, a piano, an e-piano and an organ differ greatly. Finally, the \textit{synth} family includes electronic instruments, which generate sound through analog or digital sound synthesis. Note that synthesizers do not necessarily include a keyboard but can be played via any kind of interface. All other instrument families of the proposed taxonomy are rather self-explanatory and therefore not further discussed at this point.
\figc{width=0.9\textwidth}{\pwd/figs/instrument-taxonomy-final}{Proposed two-level taxonomy: Classes with dashed lines were initially used but eventually discarded due to insufficient data.}{taxonomy}