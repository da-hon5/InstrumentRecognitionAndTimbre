%\addcontentsline{toc}{chapter}{Kurzfassung}
\begin{center}\Large\bfseries Kurzfassung
\end{center}\vspace*{1cm}\noindent
Da Musikdatenbanken und private digitale Musiksammlungen in letzter Zeit stark gewachsen sind, wurde die Verwaltung solch großer Datenmengen zu einer anspruchsvollen Aufgabe. Für verschiedene Anwendungen ist es erforderlich, Aufnahmen mit einer bestimmten Instrumentierung abrufen zu können. Darüber hinaus können auch Informationen über die Klangfarbe der einzelnen Instrumente in einem Mix hilfreich sein. Aus diesem Grund wird in dieser Arbeit ein System zur automatischen Identifizierung von Instrumenten in Musikaufnahmen, zur Bestimmung der Lautstärke jedes Instruments und zur Charakterisierung der jeweiligen Klangfarbe vorgeschlagen. Das System besteht aus zwei Hauptbestandteilen -- einem Klassifikator und einer Reihe von Klangfarbenschätzern. Der Klassifikator ist in der Lage 15 Klassen von Instrumenten aus einem Mix zu identifizieren. Anschließend bestimmen die jeweiligen Klangfarbenschätzer eine Reihe von Klangfarbendeskriptoren für jedes in dem Mix vorhandene Instrument. Alle Modelle werden als \glqq Convolutional Neural Networks\grqq{} (CNNs) realisiert, welche Mel-Spektrogramm-Repräsentationen von kurzen Ausschnitten der Musikaufnahmen als Eingangsdaten verwenden. Um auch unterrepräsentierte Klassen zum Trainieren der Modelle nutzen zu können, wurde eine zweistufige hierarchische Taxonomie erstellt, die acht Instrumentenfamilien und sieben zusätzliche spezifische Instrumente umfasst. Das Training aller Modelle wurde in zwei Phasen aufgeteilt: Nach dem Vortraining mit einem Music-Tagging-Datensatz wurden die neuronalen Netze mit drei Multi-Track-Datensätzen neu trainiert. Wir untersuchten verschiedene Transfer-Learning-Methoden und kamen zu dem Schluss, dass die besten Ergebnisse erzielt werden, wenn man eine Feinabstimmung der Convolutional Layers vornimmt. Unsere Trainingsbeispiele wurden \glqq on-the-fly\grqq{} durch das Mischen von Einzelspuren aus den Multi-Track-Daten erzeugt. Zu diesem Zweck wurden zwei verschiedene Mixing-Strategien untersucht. Es stellte sich heraus, dass eine Kombination aus beiden Mixing-Techniken am besten funktioniert. Schließlich wurden der Klassifikator und die Klangfarbenschätzer in separaten Experimenten evaluiert. Für Klassen mit ausreichend Trainingsdaten erzielt der Klassifikator F-Maße, die dem Stand der Technik entsprechen oder diesen übertreffen. Die Klangfarbenschätzer erzielten ebenfalls gute Ergebnisse; die Leistungsfähigkeit dieser Modelle ist jedoch schwer einzuschätzen, da es keine Ground Truth oder andere Arbeiten zu diesem Thema gibt.